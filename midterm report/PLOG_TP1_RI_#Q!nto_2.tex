\documentclass[a4paper]{article}

%use the english line for english reports
%usepackage[english]{babel}
\usepackage[portuguese]{babel}
\usepackage[utf8]{inputenc}
\usepackage{indentfirst}
\usepackage{graphicx}
\usepackage{verbatim}


\begin{document}

\setlength{\textwidth}{16cm}
\setlength{\textheight}{22cm}

\title{\Huge\textbf{Q!nto}\linebreak\linebreak\linebreak
\Large\textbf{Relatório Intercalar}\linebreak\linebreak
\linebreak\linebreak
\includegraphics[scale=0.1]{feup-logo.png}\linebreak\linebreak
\linebreak\linebreak
\Large{Mestrado Integrado em Engenharia Informática e Computação} \linebreak\linebreak
\Large{Programação em Lógica}\linebreak
}

\author{\textbf{Grupo 2:}\\
Filipa Marilia Monteiro Ramos - up201305378 \\
Inês Alexandra dos Santos Carneiro - up201303501 \\
\linebreak\linebreak \\
 \\ Faculdade de Engenharia da Universidade do Porto \\ Rua Roberto Frias, s\/n, 4200-465 Porto, Portugal \linebreak\linebreak\linebreak
\linebreak\linebreak\vspace{1cm}}

\maketitle
\thispagestyle{empty}

%************************************************************************************************
%************************************************************************************************

\newpage

%Todas as figuras devem ser referidas no texto. %\ref{fig:codigoFigura}
%
%%Exemplo de código para inserção de figuras
%%\begin{figure}[h!]
%%\begin{center}
%%escolher entre uma das seguintes três linhas:
%%\includegraphics[height=20cm,width=15cm]{path relativo da imagem}
%%\includegraphics[scale=0.5]{path relativo da imagem}
%%\includegraphics{path relativo da imagem}
%%\caption{legenda da figura}
%%\label{fig:codigoFigura}
%%\end{center}
%%\end{figure}
%
%
%\textit{Para escrever em itálico}
%\textbf{Para escrever em negrito}
%Para escrever em letra normal
%``Para escrever texto entre aspas''
%
%Para fazer parágrafo, deixar uma linha em branco.
%
%Como fazer bullet points:
%\begin{itemize}
	%\item Item1
	%\item Item2
%\end{itemize}
%
%Como enumerar itens:
%\begin{enumerate}
	%\item Item 1
	%\item Item 2
%\end{enumerate}
%
%\begin{quote}``Isto é uma citação''\end{quote}


%%%%%%%%%%%%%%%%%%%%%%%%%%
\section{O Jogo Q!nto}

\begin{figure}[ht!]
\centering
\includegraphics[width=90mm]{q!ntologo.jpg}
\caption{Caixa e cartas do jog. \label{q!nto}}
\end{figure}

\subsection{História}

Q!nto é um jogo de estratégia abstrata que foi desenvolvido pelo designer Gene Mackles e publicado pela PDG games. O seu lançamento no mercado data do ano 2014. É adequado para todas as idades a partir dos 8 anos podendo ser jogado por um mínimo de 2 e um máximo de 4 jogadores. Existem 3 variações de Q!nto: Q!nto clássico que será o desenvolvido pelo grupo; Q!nto Plus que permite a contagem de pontos numa diagonal de 3 ou mais cartas e Q!nto Light no qual as cartas são divididas igualmente pelos jogadores e o vencedor é o que esvazia a sua mão primeiro. 

\subsection{Regras}

Cada jogador possui 5 cartas sendo que existem 5 formas e 5 cores possíveis. Existem cartas, em menor número, que permitem escolher ou a sua forma ou a sua cor ou ambas.

\begin{enumerate}
	\item Carta que permite escolher a cor: tem uma forma específica e o utilizador escolhe a cor que esta possuí. 
	\item Carta que permite escolher a forma: tem uma cor definida e permite a escolha da forma da mesma.
	\item Carta universal: forma e cor definível. 
\end{enumerate}

\begin{figure}
\centering
\begin{subfigure}{.5\textwidth}
  \centering
  \includegraphics[width=.4\linewidth]{corEscolhivel.jpg}
  \caption{A subfigure}
  \label{fig:sub1}
\end{subfigure}%
\begin{subfigure}{.5\textwidth}
  \centering
  \includegraphics[width=.4\linewidth]{formaEscolhivel.jpg}
  \caption{A subfigure}
  \label{fig:sub2}
\end{subfigure}
\begin{subfigure}{.5\textwidth}
  \centering
  \includegraphics[width=.4\linewidth]{todas.jpg}
  \caption{A subfigure}
  \label{fig:sub3}
\end{subfigure}%
\caption{A figure with two subfigures}
\label{fig:test}
\end{figure}



%\begin{figure}[]
%\raggedright
%\includegraphics[width=20mm]{corEscolhivel.jpg}
%\caption{Carta que permite escolher a cor (1). \label{corEsc}}
%\end{figure}
%
%\begin{figure}
%\centering
%\includegraphics[width=20mm]{formaEscolhivel.jpg}
%\caption{Carta que permite escolher a forma (2). \label{formaEsc}}
%\end{figure}
%
%\begin{figure}
%\raggedright
%\includegraphics[width=20mm]{todas.jpg}
%\caption{Carta universal (3). \label{Univ}}
%\end{figure}

Uma jogada é válida se for feita uma coluna ou linha com cartas nas seguintes condições:

\begin{enumerate}
	\item símbolos iguais;
	\item cores iguais;
	\item símbolos e cores diferentes.
\end{enumerate}

A pontuação é calculada com base no número de cartas por linha e coluna englobada na jogada. A cada carta é dada a pontuação de uma unidade, sendo que quando se obtêm linhas com 5 elementos o jogador recebe uma pontuação extra de 5 pontos. 

%%%%%%%%%%%%%%%%%%%%%%%%%%
\section{Representação do Estado do Jogo}

O tabuleiro inicial corresponde a uma grelha de 60x60 visto que existem 60 cartas no total. Não existem posições iniciais pois trata-se de um jogo de cartas. Por baixo do tabuleiro é representada a mão do jogador. 

%%%%%%%%%%%%%%%%%%%%%%%%%%
\section{Visualização do Tabuleiro}

A representação do tabuleiro será feita de forma textual parecendo-se com o seguinte: \linebreak ---- ---- ---- \linebreak
| r* |    |    | \linebreak
|----|----|----| \linebreak
|    |    |    | \linebreak
|----|----|----| \linebreak
|    |    |    | \linebreak
 ---- ---- ----  \linebreak
\linebreak


%%%%%%%%%%%%%%%%%%%%%%%%%%
\section{Movimentos}

Elencar os movimentos (tipos de jogadas) possíveis e definir os cabeçalhos dos predicados que serão utilizados (ainda não precisam de estar implementados).


\end{document}